\section{Persönliche Meinung}
\label{sec:persoenliche_meinung}

Um einen besseren Eindruck von dem bearbeiteten Artikel zu vermitteln, soll schlussendlich noch meine persönliche Meinung zu der Vorgehensweise der Autoren genannt werden.
Allgemein finde ich, dass der Text einsteigerfreundlich geschrieben ist, sodass er auch für Leser, die nicht besonders tief mit der Materie vertraut sind, leicht verständlich sein sollte.
Abgesehen davon werden aber auch nicht zu viele Grundlagen aus dem Themenbereich wiederholt, sodass sich auch erfahrene Leser nicht langweilen dürften.

Außerdem wirkt der Text allgemein wissenschaftlich fundiert, wozu auch der im Anhang des Originals stehende Beweis beiträgt.
Besonders überzeugend ist auch, dass die Autoren in dem nachfolgend veröffentlichten Artikel \cite{WangJFP16} weitere Tests in unterschiedlichen Szenarien durchführen, welche die Ergebnisse weiter untermauern.

Weniger gut gefällt mir, dass die Autoren einige für mich offensichtliche Auffälligkeiten teilweise überhaupt nicht erklären.
Beispielsweise wäre hier zu nennen, dass in Abbildung \ref{fig:commit_abort} d die Transaktionsabbrüche bei Verwendung der SSI mit zunehmender Thread-Zahl rasant steigt, ab 18 Threads aber wieder langsam abnimmt.
Dieses Verhalten kann ich mir absolut nicht erklären und auch im Artikel wird darauf nicht weiter eingegangen.

Des weiteren wird in dem Artikel nicht wirklich eine klare Begründung für die schlechte Performanz der Serializable-Snapshot-Isolation geliefert.
Es wird zwar an einer Stelle gesagt, dass es etwas mit einer hohen Zahl von Indexoperationen zutun hat, allerdings wird darauf nicht weiter eingegangen.
In dem später erschienenen Artikel \cite{WangJFP16} werden hier zwar einige weitere Erklärungen gegeben, allerdings fehlen dort meiner Meinung nach einige Hinweise in dem Originalartikel \cite{Wang:2015}.

Als weitere Begründung für die schlechte Performanz von SSI wird genannt, dass eine hohe Zahl von Transaktionsabbrüchen durch den sogenannten Temporal Skew verursacht werden.
Dabei schreibt eine Transaktion auf einen Wert, der nach seinem Snapshot überschrieben wurde. 
Dieser Fall tritt allerdings bei der normalen SI ebenfalls auf, weshalb dies nicht erklärt, warum SI+SSN an den meisten Stellen deutlich besser abschneidet als die SSI.

Bei der Erklärung von Abbildung \ref{fig:commit_abort} d wurde vorher begründet, dass SSI eine besonders geringe Abbruchrate besitzt, da die allgemeine Performanz so schlecht ist, dass nur wenige Transaktionen pro Sekunde bearbeitet werden, was natürlich zu einer geringen Zahl von Transaktionsabbrüchen führt.
Der gleiche Effekt müsste aber doch auch bei den anderen Verfahren in irgendeiner Form existieren, weil diese ja auch alle einen unterschiedlichen Overhead produzieren, wodurch die Aussagekraft der Ergebnisse, die in Abbildung \ref{fig:commit_abort} dargestellt wurden, meiner Meinung nach etwas eingeschränkt ist.